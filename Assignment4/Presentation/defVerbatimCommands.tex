\defverbatim[colored]\ioBaselineCodeRead{%
\begin{lstlisting}[basicstyle=\tiny\ttfamily]
if (rank == 0){
   int row, column;
   if ((fp = fopen (argv[1], "r")) != NULL){
       fscanf(fp, "%d %d\n", &matrices_a_b_dimensions[0], &matrices_a_b_dimensions[1]);
       A = (double **) malloc (matrices_a_b_dimensions[0] * sizeof(double *));
       for (row = 0; row < matrices_a_b_dimensions[0]; row++){
            A[row] = (double *) malloc(matrices_a_b_dimensions[1] * sizeof(double));
            for (column = 0; column < matrices_a_b_dimensions[1]; column++)
                fscanf(fp, "%lf", &A[row][column]);
        }
        fclose(fp);
    } else {
        if(rank == 0) fprintf(stderr, "error opening file for matrix A (%s)\n", argv[1]);
        MPI_Abort(MPI_COMM_WORLD, -1);
    }
    /* Here same for B matrix */

    // need to check that the multiplication is possible given dimensions 
    /* Checks for right dimensions */
}
\end{lstlisting}}%

\defverbatim[colored]\ioMPICodeRead{%
\begin{lstlisting}[basicstyle=\tiny\ttfamily]
MPI_Info readInfo;
MPI_Info_create(&readInfo);
int ierr;
int blocksize_A[2] = {A_rows, A_columns*characters_per_number};
MPI_Datatype datatype_blocks_A;
int subsize_A[2] = {A_local_block_rows, A_local_block_columns*characters_per_number};
int array_of_starts_A[2] = {coordinates[0]*A_local_block_rows,((coordinates[1] + coordinates[0]) % sqrt_size)*A_local_block_columns*characters_per_number};    //This shifts the starting coordinate according to the original shift in Cannon's algorithm!
ierr = MPI_Type_create_subarray(2, blocksize_A, subsize_A, array_of_starts_A, MPI_ORDER_C, MPI_CHAR, &datatype_blocks_A);          
MPI_Type_commit(&datatype_blocks_A);
MPI_File mpi_file_matrixA;
ierr=MPI_File_open(MPI_COMM_WORLD, argv[1], MPI_MODE_RDONLY | MPI_MODE_UNIQUE_OPEN, MPI_INFO_NULL, &mpi_file_matrixA);
if (ierr!=MPI_SUCCESS) {printf(" Cannot open file\n");}
MPI_File_set_view(mpi_file_matrixA, A_file_header_size, MPI_CHAR, datatype_blocks_A, "native", readInfo);
	
MPI_File_read_all(mpi_file_matrixA, A_local_block_read, A_local_block_size * characters_per_number, MPI_CHAR, MPI_STATUS_IGNORE);
		
MPI_File_close(&mpi_file_matrixA);
\end{lstlisting}}%

\defverbatim[colored]\ioMPICodeWrite{%
\begin{lstlisting}[basicstyle=\tiny\ttfamily]
int blocksize_C[2] = {A_rows, B_columns};
MPI_Datatype datatype_blocks_C;
int subsize_C[2] = {A_local_block_rows, B_local_block_columns};
int array_of_starts_C[2] = {coordinates[0]*A_local_block_rows, coordinates[1]*B_local_block_columns};    //This shifts the starting coordinate according to the original shift in Cannon's algorithm!
                           
ierr = MPI_Type_create_subarray(2, blocksize_C, subsize_C, array_of_starts_C, MPI_ORDER_C, MPI_DOUBLE, &datatype_blocks_C);
	
MPI_Type_commit(&datatype_blocks_C);
MPI_File mpi_file_matrixC;
ierr=MPI_File_open(MPI_COMM_WORLD, output_filename, MPI_MODE_WRONLY | MPI_MODE_CREATE | MPI_MODE_UNIQUE_OPEN, MPI_INFO_NULL, &mpi_file_matrixC);
if (ierr!=MPI_SUCCESS) {printf(" Cannot open file\n");} 
MPI_File_set_view(mpi_file_matrixC, 0, MPI_DOUBLE, datatype_blocks_C, "native", writeInfo);
	
MPI_File_write_all(mpi_file_matrixC, C_local_block, C_local_block_size, MPI_DOUBLE, MPI_STATUS_IGNORE);
		
MPI_File_close(&mpi_file_matrixC);
\end{lstlisting}}%