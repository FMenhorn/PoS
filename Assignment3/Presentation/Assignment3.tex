\documentclass[9pt,pdftex]{beamer}
\setbeamertemplate{section in toc}[sections numbered]
\setbeamertemplate{subsection in toc}%
{\leavevmode\leftskip=3em\rlap{\hskip-2em\inserttocsectionnumber.\inserttocsubsectionnumber}\inserttocsubsection\par}
% use git: import repository as new project in eclipse: http://www.eclipse.org/forums/index.php/t/226301/
\usepackage[utf8]{inputenc}
\usepackage[english]{babel}
\usepackage{amsfonts, amsmath, amssymb}
\usepackage[bf,small, format=plain]{caption}
\usepackage{color}

%\usepackage[usenames,dvipsnames]{xcolor}
\usepackage{graphicx}
\usepackage{tikz,tikzscale,pgfplots,grffile}
\usetikzlibrary{arrows,shapes,backgrounds}
\usetikzlibrary{plotmarks}
\usepackage{bbding}

\usepackage[clock]{ifsym}

\usepackage{pgfplots}
\usepackage{grffile}
\pgfplotsset{compat=newest}

%\author{}
%\title{}
%\setbeamercovered{transparent} 
%\setbeamertemplate{navigation symbols}{} 
%\logo{} 
%\institute{} 
%\date{} 
%\subject{} 

\pgfdeclarelayer{bg}
\pgfsetlayers{bg,main}

\usepackage{listings}
\lstset{language=C++,
				tabsize=8,
				showtabs=true,
                basicstyle=\ttfamily,
                keywordstyle=\color{blue}\ttfamily,
                stringstyle=\color{red}\ttfamily,
                commentstyle=\color{green}\ttfamily,
                morecomment=[l][\color{magenta}]{\#}
}
\usepackage{mdframed}
\usepackage{multicol}
\usepackage{tcolorbox}
%\usepackage{multirow}
% \usepackage{paralist}
%\usepackage[colorinlistoftodos]{todonotes}
%\usepackage{biblatex}
%usepackage[nolist,nohyperlinks]{acronym}
%\usepackage{amstext}
%\usepackage{hyperref} 
%\usepackage{comment}
%\usepackage{subcaption}
%\renewcommand*{\figureautorefname}{fig.}
%\renewcommand*{\equationautorefname}{eq.}
\usepackage{bm}
\usepackage{comment}
%\usepackage{beamerthemeshadow}
%\usepackage{tikz}
%\usepackage{pgfplots}
%\usepackage{ulem}
%\usepackage[lofdepth,lotdepth]{subfig}
%\newenvironment{figure*}%
%{\begin{figure}}
%{\end{figure}}
%\usepackage[style=mla,babel=hyphen,backend=biber]{biblatex}
% CSE-Beamer-Styles:
\usepackage[course]{beamertheme_sccstalk}
%\usepackage[lecture]{beamertheme_sccstalk}
\usepackage{beamercolorscheme_sccs}
\usepackage{beamerfontthemestructurebold}
\setcounter{tocdepth}{3} 
%colored blocks, example \begin{variableblock}{Title}{bg=blue,fg=white}{bg=white,fg=black}
\newenvironment{variableblock}[4]{%
\setbeamercolor{block title}{#2}
\setbeamercolor{block body}{#3}
\begin{block}{#1}\begin{mdframed}{#4}\end{mdframed}\end{block}}


%some useful commands
%\newcommand{\der}[2]{\frac{\text{d}#1}{\text{d}#2}}
\title{Assignment 3: MPI Point-to-Point and One-Sided Communication}
\subtitle{Programming of Super Computers}
\author[Friedrich Menhorn, Benjamin Rüth, Erik Wannerberg] {Friedrich Menhorn, Benjamin Rüth, Erik Wannerberg \\ Team 12} %[displayed in footer]{displayed on title page}
\date{\today}
\institute{Technische Universität München}
\newtheorem*{rem}{Remark}

\begin{document}
\frame{\maketitle}

\begin{frame}{Contents}
\tableofcontents
\end{frame}

\section{Provided Implementation and Baseline}
\begin{frame}{\phantom{Contents}}
\tableofcontents[
  currentsection  
]
\end{frame}

\subsection{Cannon’s algorithm}
\begin{frame}{\insertsubsection}
\begin{itemize}
\item explain algorithm
\item provided implementation
\end{itemize}
\end{frame}


\subsection{Baseline}
\begin{frame}{\insertsubsection}
Challenges in getting an accurate baseline and changes to the Load-Leveler batch script.
\end{frame}

\subsection{Scalability}
\begin{frame}{\insertsubsection}
\begin{itemize}
\item Compute time scalability with fixed 64 processes and varying size of input files.
\item MPI time scalability with fixed 64 processes and varying size of input files.
\item Differences in scalability between the Sandy Bridge and Haswell architectures.
\end{itemize}
\end{frame}

\section{MPI Point-to-Point Communication}
\begin{frame}{\phantom{Contents}}
\tableofcontents[
  currentsection  
]
\end{frame}
\subsection{MPI Non-Blocking Operations}
\begin{frame}{\insertsubsection}
\begin{itemize}
\item Which non-blocking operations were used?
\end{itemize}
\end{frame}
\subsection{Optimizations}
\begin{frame}{\insertsubsection}
\begin{itemize}
\item Was communication and computation overlap achieved?
\item What is the theoretical maximum overlap that can be achieved? Explain.
\end{itemize}
\end{frame}
\subsection{Scaling}
\begin{frame}{\insertsubsection}
\begin{itemize}
\item Was a speedup observed versus the baseline?
\item Were there any differences between Sandy Bridge and Haswell nodes?
\end{itemize}
\end{frame}

\section{MPI One-Sided Communication}
\begin{frame}{\phantom{Contents}}
\tableofcontents[
  currentsection  
]
\end{frame}
\subsection{MPI One-Sided Operations}
\begin{frame}{\insertsubsection}
\begin{itemize}
\item Which one-sided operations were used?
\end{itemize}
\end{frame}
\subsection{Optimizations}
\begin{frame}{\insertsubsection}
\begin{itemize}
\item Was communication and computation overlap achieved?
\end{itemize}
\end{frame}

\subsection{Scaling}
\begin{frame}{\insertsubsection}
\begin{itemize}
\item Was a speedup observed versus the baseline?
\item Was a speedup observed versus the non-blocking version?
\item Were there any differences between Sandy Bridge and Haswell nodes?
\end{itemize}
\end{frame}

\end{document}
